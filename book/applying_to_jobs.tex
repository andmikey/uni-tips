\chapter{Applying to jobs}
\newcommand{\email}[1]{
\noindent\fbox{%
    \parbox{\textwidth}{%
	#1
    }%
}
}

\section{CVs}

\section{Cover letters}

\section{Interviews}
\subsection{Modes of interview}
\subsubsection{Phone interview}
\subsubsection{Video interview}
\subsubsection{In-person interview (onsite)}
\subsubsection{Assessment centre}

\subsection{Types of interview}
\subsubsection{Screening call}
\subsubsection{Coding challenge}
\subsubsection{Take-home project}
\subsubsection{Pre-recorded video interview}
\subsubsection{Online assessment}
\subsubsection{Competency-based interview}
\subsubsection{Technical interview}
\subsubsection{Whiteboarding interview}

\subsection{Interviews as a two-way street}

It's important to remember that an interview isn't just about the company seeing if you're a good fit for the role. It's also an opportunity for you to assess whether the role is a good fit {\it for you}. Treat it as such! 

\subsubsection{Questions to ask the interviewer}
At the end of an interview, the interviewer will ask you ``do you have any questions for me?''. The correct answer to this is not ``no''. This is an opportunity for you to show your enthusiasm for the team's work --- and more importantly, to ask questions about things that are important to you. 

Questions you could ask include:

\begin{itemize}
\item At the end of this internship, what skills do you want the intern to have learned or developed? What processes or training do you have in place to ensure that? \\{\it This lets you see how much they care about what {\it you} get out of the internship, and how they help you achieve that.}
\item (If interviewing with your future manager) What is your approach to managing interns? How often would I have one-to-one time with you? \\{\it As an intern, it's important you get mentorship from someone senior to you --- you're checking here if they will actually have time to give you advice.}
\item What project do you have planned for the intern to work on, if any? What do you think the most challenging part of it will be? \\{\it Very good question to ask if you're interviewing with the team you'll be working on --- it's good to know what you'll actually be working on for the summer! Also good to watch out for is whether this is an `independent' project or whether you'll be working with other team members to complete it.}
\item What impact do you expect the intern's work to have on the team? \\{\it Key thing to watch for: will you be given tasks that matter to the team's work, or shunted into a corner and given grunt work to keep you busy? It's also a good signal to the interviewer that you care about the work you're doing.}
\item Have you had interns before? What projects did they work on? Are those projects still being used now? \\{\it This is a more detailed version of the previous question that forces them to give you a concrete example of whether previous interns' work is actually used within the team. }
\item What practices do you use to ensure code quality within your team? \\{\it The answer to this should include: version control, unit tests, coding standards, code reviews. If i doesn't, you're likely to pick up bad habits while working here.}
\item What is the most challenging or interesting project your team is working on at the moment? \\{\it This will give you a broader insight into the kind of work the team is doing right now, as well as give them a chance to show off their `coolest' work. }
\item What is your favorite part of working for this team and this company? \\{\it If they struggle to answer this, tread very carefully.}
\item How does your team fit in with the rest of the company? What other teams do you regularly interact with? \\{\it It's good to get experience in a team that's `in on the action' rather than working isolated from the rest of the company. This question can also help you understand what other parts of the business you might interact with.}
\end{itemize}

\subsubsection{Things to watch out for}
Just like you should present your best self in the interview, so should the people you're interviewing with. 

\begin{itemize}
\item If you're interviewing with multiple people, what are the dynamics between them? Are they arguing or disagreeing with each other? \\{\it If there is conflict between your interviewers during the interview, it's likely that conflict is even more present in the workplace. }
\item How does the person you're interviewing with speak about others (coworkers, previous interns, people from other departments)? \\{\it If the interviewer is negative about others, they'll likely be negative about you as well. }
\item What is the tone of the person asking you questions? Are they dismissive or condescending, particularly if you don't know the answer to a question? \\{\it You don't want to work with someone who isn't enthusiastic about helping you learn and improve your knowledge; and likewise you don't want to work with someone who you don't feel comfortable approaching when you have a question.}
\end{itemize}

\section{Hearing back}

\subsection{Offers}

Congratulations! You've conquered the interview gauntlet and received an offer. 

\subsubsection{Accepting an offer}

This is the easiest step! If the offer is a good fit, then all you need to do is accept it. 

\email{
	Dear {\it RecruiterName},
	
	Thank you for your email. I am very happy to accept this offer. 
	
	Could you let me know what the next steps will be?
	
	Kind regards,
	
	{\it YourName}
}

After you send that email, relax! You're done for the recruitment cycle. You'll receive a contract to sign confirming the dates of the internship, your role, and your pay. You're done! 

\subsubsection{Rejecting an offer}

Sometimes it happens that you don't want to take up an offer. This could be for many reasons, including:
\begin{itemize}
\item Having received a better offer
\item The timing being awkward (maybe it overlaps with term time, maybe it covers your entire summer holiday and you don't want to work for that long)
\item The pay being below what you expect / require
\item The role not fitting your career goals
\item The location not fitting your plans for the summer
\end{itemize}

That's fine! Companies don't expect everyone to accept their offers. 

You can reject an offer by simply sending a polite email to the recruiter:

\email{
Dear {\it RecruiterName},

Thank you for your email. I am glad to hear I have received an offer for this position.

Unfortunately, I have chosen to move forward with other opportunities, so I will  have to decline this offer.

Thank you once again for your time and help throughout this process. I would love to keep in touch about future opportunities at {\it CompanyName}. 

Kind regards,

{\it YourName}
}


\subsubsection{Receiving an offer while interviewing elsewhere}

Chances are you'll be interviewing for multiple positions at once, and not all of those positions will get back to you at the same time. You may hear back with an offer from one company before you finish interviews (or hear back) from another. 

If you're happy with this offer and you would be with this position --- great! Email your recruiter back with an acceptance, and withdraw from your existing interview processes. 

If you're not sure about this offer --- that is, you want to wait until you've heard back from the other places you're interviewing before you make a decision --- then you need to go from two fronts: firstly, contact the company that made you an offer and ask if you can delay accepting the offer until you have heard back from all companies. Secondly, contact the companies you're still interviewing with and ask if they can accelerate the interview processes to meet the other company's deadline. 

To request more time from the company that has given you an offer:

\email{
	Dear {\it RecruiterName},
	
	Thank you for your email. I very much enjoyed interviewing with {\it CompanyName} and I am happy to hear I have received an offer for this position.
	
	However, I am still waiting on replies from other companies I am interviewing with. While {\it CompanyName} is my first choice, I would like to complete my existing interview processes before making a final decision.
	
	Would it be possible to extend the deadline on me making a decision until {\it Date}?
	
	Kind regards,
	
	{\it YourName}
	}
	
To accelerate an ongoing interview process:

\email{
	Dear {\it RecruiterName},
	
	I am writing to you regarding my application for the position of {\it PositionName}. I have received an offer from another company which expires on {\it Date}.
	 
	Could you let me know if it will be possible to accelerate the interview process so that I can make my decision before {\it Date}?
	
	Kind regards,
	
	{\it YourName}
}

To ask the company to hurry up on giving you a decision after a final interview: 

\email{
	Dear {\it RecruiterName},
	
	Following my interview on {\it Date} I would like to ask when I can expect to hear back regarding my candidacy for this role. I have received an offer from another company which expires on {\it Date} and I am keen to hear back from {\it CompanyName} before I make a final decision.
	
	Kind regards,
	
	{\it YourName}
}

If one side or another won't budge, you may have a hard decision to make. Do you reject the offer and hope that you get an offer from the other companies you're interviewing with --- or do you accept the offer and withdraw from your existing interview processes? How you approach this is up to you, but here's a few things to think about:
\begin{itemize}
\item How likely is it that you'll get an offer from the places you're still waiting to hear back from? Are you confident about your performance so far?
\item If you reject the existing offer and you don't get any other offers, how would you feel about that? Do you have backups in place (e.g. other internships to apply for, other things to do during the summer)?
\item How do you feel about the offer you've received? Is it a good step, career-wise? Will it give you an opportunity to learn the skills you want to develop? How do the other positions you're interviewing for compare? Are they significantly better?
\end{itemize}

\subsection{Rejections}
\subsubsection{With feedback}
\subsubsection{Without feedback} 
\subsubsection{Ghosting}

\subsection{Withdrawing an application}

\subsection{Reneging} 